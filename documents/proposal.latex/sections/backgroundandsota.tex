%!TEX root = ../proposal.tex
\section{Background and State of Art}
Lot of work is being done to enhance the automation of self driving cars. Here i will describe the related work to specific areas that focus on the goal of my thesis.  

\subsection{Natural Language Processing and Ontology}

Part of Speech (POS Tag)\cite{marquez1998part} is a technique relies on Natural Language Processing \cite{loper2002nltk} to extract the key features \cite{mitkov2007anaphora} such as Noun Phrase, Verb Phrase to understand the context and meaning of the sentence. Guarino et al \cite{guarino2009ontology} illustrate ontology's as “explicit specification of a shared conceptualization.” The ontology's \cite{studer1998knowledge} consist of names, definition, properties and inter-relationship of the particular domain like automotive application and weather application. Armand et al \cite{armand2014ontology} depict that how ontology's can be applied to Advanced Driving Assistance System (ADAS) to model the spatial-temporal relationship between the cars. The classified entities in ontology’s to understand map, vehicle (Mobile Entity), track (Static Entity) and dynamic state (Context Parameter) of the ego vehicle. Hummel el al\cite{hummel2008scene} enlightened the description logic to find the intersection and geometry of roads. Similar approaches to scene understanding for ADAS by Zhao et al \cite{zhao2015ontology} to obtain information about traffic, vehicle and and street map.

\subsection{Open Street Map (OSM)}
Godoy et al\cite{godoy2019self} come up with procedure to identify the road intersections and junctions in OpenStreetMap \cite{haklay2008openstreetmap}. The path were extracted from the Open Street Map based on the key attribute such as road type, surrounding area and scenic beauty. Guillaume et al \cite{bresson2017simultaneous} surveyed the challenges in Simultaneous Localization And Mapping (SLAM) for the localization of driving cars in Enhanced Maps to resist the environment variability (weather, season). The paper provide an overview of the challenges in large scale experiments of autonomous cars.  

\subsection{Automatic Test Case Generation}

In  Erbsmehl \cite{erbsmehl2009simulation},  the  author  focus  on  the  simulation  of  advanced  safety  systems  on  real  world crashes based on German In-Depth Accident Study database.  The comparative and analysis study is done on simulated real world accident scenario and a predicted accident scenario on virtual prototype of safety system to estimate the benefit of safety system in vehicles. The GIDAS pre crash matrix contains all the relevant information such as the geometrical information of the accident (driving lane of each participants, lane borders, lane markers, view obstacles) and the dynamic behaviour of the participants. CarSim with Matlab is used for the simulation each participant and vehicle dynamic with additional car parameter, physical parameters is calculated at every 0.001 seconds. Squillant et al\cite{squillante2018modeling} focus  on  the  modeling  of  accident  scenarios  with  missing data  in  the databases and the design of safety systems related to the accident. Zhou et al\cite{zhou2018safety},  the author proposes a Gaussian Process Classification algorithm to efficiently determine the non-convex boundaries based on the limited number of simulations. The modelling of the safety boundary search consist of a test case ”yaw” with 3 dimensional inputs such as longitudinal inter-vehicle distance,the longitudinal velocity difference between ego vehicle and cut-in vehicle,  cut-in angle yaw of the cut-in vehicle and and binary outputs (Collision or not Collision). \\

Arrieta et al\cite{arrieta2018employing} studied the traditional testing techniques,
Model based testing are not feasible to apply to Cyber Physical Systems because it is not feasible to evaluate all the possible state and values, So the simulation models with search-based approach are developing to execute large number of reactive test cases with prioritization to effectively execute test cases with critical scenarios and replication of safety critical functions. The fitness function with four objectives functions are  analyzed  with  crossover  and  mutation  operator  incorporating  five  multi-objective
search algorithms is evaluated in case studies to find the optimal solution. Furthermore, the modules of 3D simulation to demonstrate the Scene Graph Model and validate the trajectory and autonomous component. The number of traffic scenarios can be evaluated by model parameterization to increase the coverage of testing space\cite{zofka2016testing}. 

%\subsection{Edge Detection}
%Xiaohui et al\cite{zhan2007improved} present an approach to find the moving object in the frame and detect the object with reference to edge detection and frame difference algorithm. It split the image into small parts and find the disparity between each block to speed up the process of edge detection. Li et al\cite{li2018key} extract keyframe from videos and present the summary based on the content of the video and eliminate redundant frames. The comparison of the extracted images is measured with Structural Similarity Index Measure whether there is a match for image processing of content retrieval\cite{Premaratne2012NewSS}.    

\subsection{Decision Tree}
The series of simulation output can be visualized and explicitly shown on Decision Tree for decision analysis\cite{abdessalem2018testing}. The decisions tress express the next parameter input in the complex multidimensional input parameter space. The test cases generated with complex scenario is used to categorize the critical scenario in Advanced Driving Assistance System (ADAS). It will further help in reducing the parameter space and choose the best parameter to identify parameter and environmental conditions. \
The decision tree will not be used to predict the output of the parameter using machine learning technique whether the scenario is critical or not. The decision tree will be exclusively use to (1) better guide the search and limiting the parameter space, and (2) to characterize
the critical regions of the parameter input space. Furthermore, the stopping criterion is defined to avoid over-fitting and to stop the tree expansion such that depth of tree and number of descendant does not fall below a certain threshold. 
